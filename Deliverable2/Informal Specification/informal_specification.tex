\documentclass[12pt]{article}
\usepackage{ltcadiz}
\usepackage{listings}
\usepackage{amssymb}
\lstset{breaklines=true, numbers=left}

\title{Report on OGO 2.2 Softwarespecification\\ Assignment 2a}
\author{
        Tim van Dalen, Tony Nan, Ferry Timmers, \\ Lasse Blaauwbroek, Femke Jansen, \\Jeroen Peters and Sander Breukink\\ OGO 2.2 group 6 \\
                Department of Computer Science\\
        Technical University Eindhoven\\
}
\date{\today}

\begin{document}

\maketitle

\begin{abstract}
\end{abstract}

\section{Game}

Inleiding dolfijnen en vossen.

\subsection{Playing field}

The playing field consists of a two-dimensional grid of hexagonal tiles. Every tile has an associated elevation value which can be represented using color. Some of the tiles are flooded with water (depending on their elevation) and the other tiles are land. Initially, the tiles with land have to be adjacent. Two adjacent tiles (meaning they have a common edge) may only differ in two units of elevation. We can have bridges and underwater caves between two tiles. A dolphin can swim across a cave as long as both endpoints are flooded and a fox can cross a bridge as long as both endpoints are not flooded.

The elevation of a tile lies between minus fifty and fifty. The water elevation lies between minus fifteen and fifteen. The sum of the elevation of all tiles must be zero. The tide of the water determines the exact elevation. The tide can randomly change between rounds and changes between one and five elevation units. The speed of propagation of water is two tiles per round (in the natural direction of the propagation).

\subsection{Gameplay}
There are two players, each of them controlling a set of dolphins or a set of foxes. Initially there are an equal number of foxes and dolphins. This number can be defined beforehand by the players. The size of the playing board is dependent on the initial number of animals.

The game is played in rounds. In every round each player can move one quarter of the total number of animals at that moment in the game. A move means that one animal can move from one tile to an adjacent tile (or cross a bridge or cave). The player can determine how he distributes his moves between his animals. After a round the computer can adjust the tide level and let the water propagate accordingly.

A fox in the water can be eaten by a dolphin on the same tile. The fox vanishes from the board when this happens. When a fox is further than two tiles away from the land it cannot move until the tide drops again. Similarly, a dolphin on the land can be eaten by a fox and if it is more than two tiles away from the water it cannot move until the tide rises again.

The game has ended when one of the players has no animals left.

\subsection{Natural events}
Between two rounds a natural event can occur. Those events are the dry season and the wet season. These seasons can shift the maximum and minimum value of the tide with a maximum of twenty units. Dry season means the shift is negative and wet season means the shift is positive. A season lasts for five rounds. A natural event occurs with a probability of five percent. And no two natural events can occur a the same time.

\end{document}

\lstset{
	tabsize=5,
	basicstyle=\small,
}

\begin{lstlisting}
Als een robot die meerdere hokjes in een keer kan afgleggen over een hint/
thuis/lopende band heen gaat, wat gebeurt er dan?
	Hint niets.
	Thuis wint.
	Lopende band gesleurd.
Als een hintvlak maar een hint kan geven, hoe kan hij dan een volgende 
keer een andere hint geven?
	Random.
Als een robot naar een vakje probeert te gaan wat bezet is, kan hij dan 
in dezelfde beurt nog een andere zet doen?
	Geen beurten, geen probleem.
Als een robot ergens op de lopende band terecht komt (niet begin / einde), 
wordt hij dan getransporteerd?
	Meegesleurd naar het einde.
Zijn de transportiebanden alleen een begin en eindvlakje? En als een robot 
nou aan het einde door robots is ingesloten, kan deze dan terug de lopende 
band op en verandert deze dan van richting?
	Aan het eind van een lopende band moet je je thuisvlak kunnen bereiken, 
	lopende banden zijn 1 kant op.
Hoe ver wordt een robot verplaatst als deze op een transportatieband 
(geen rotatie) terecht komt? 
	In 1x naar het einde.
Wat gebeurt er als het einde van de lopende band is ingesloten door 
defecte robots?
	-
Wat gebeurt er als er een defecte robot op een lopende band staat? 
Blijft deze staan of wordt hij getransporteerd?
	Je wisselt vakjes om, dus de lopende band stopt daar gewoon.
Kan een defecte robot ook een thuis/hintvlak blokkeren?
	Aantal soorten vakjes, die niet kunnen overlappen.
Kan een hintvlakje een of twee richtingen als hint geven?
	-
Kunnen transportatiebanden zowel rotatie als transport zijn?
	Beide, lopende band verdraaid random je rotatie.
Wat gebeurt er als een robot in het midden van een lopende band terecht komt, 
of zijn er alleen opstap en eindpunten?
	Gewoon naar het einde.
De twee robots hebben eigenlijk precies dezelfde functionaliteit aangezien er 
geen rondes zijn, en de robots dus zoveel moves in een tijdsunit mogen doen 
als ze willen. Daarom kan robot A net zo snel bewegen als robot B. En omdat 
robot B niet springt moet hij over alle vakjes heen en is het netto effect 
hetzelfde.
	Gewoon implementeren, het resultaat mag hetzelfde zijn. Het enige verschil 
	is dat robot B in zijn weg een hint tegen kan komen waar hij overheen 
	loopt. De robot krijgt deze hint dan niet te zien.

\end{lstlisting}

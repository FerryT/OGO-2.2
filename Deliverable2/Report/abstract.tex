\begin{abstract}

For the informal specification given in "Informele Specificatie" by OGO 2.2 group 2 (see appendix~\ref{appendix:informal}), a formal specification will be introduced here. It will formalize the core points of the given informal specification in order to remove all ambiguity. The formal specification is, as in the informal specification, about a game played by robots.

First of all, questions to the stakeholder will be given. Since the informal specification was not totally clear on every point, we asked some questions to the stakeholders.
Secondly, some use cases will be introduced. In addition to this, a diagram will be given which shows how the use cases work together. These use cases specify some functionality of the game.
After this, the message sequence charts will be shown. These are made to explain which messages are send to deal with every use case. A high level message sequence chart, which is made to show how the message sequence chart work together, will also be shown here.
Then, the class diagram and class descriptions will be given. Here, all the classes will be shown along with their descriptions.
The state charts come next. They are made to show the different states of the classes and what triggers the state changes in these classes.
Finally, a specification in Object-Z will be given. This should show the functionality of the game, what is possible and what not. In the appendix, the informal specification is provided.

\end{abstract}
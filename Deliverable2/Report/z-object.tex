\documentclass[12pt]{article}
\usepackage{ltcadiz}
\usepackage{listings, a4wide}
\lstset{breaklines=true, numbers=left}

\title{Report on OGO 2.2 Softwarespecification\\ Assignment 1a}
\author{
        Femke Jansen and Lasse Blaauwbroek OGO 2.2 group 6 \\
                Department of Computer Science\\
        Technical University Eindhoven\\
}
\date{\today}

\begin{document}

\maketitle

\begin{abstract}

\end{abstract}

\section{Basic Axiomatic definitions}
We define several attributes that we will use in later definitions in our Z schemas. Since some returned objects in our class diagram may contain null values, we have defined it as a separate set. \\
In our class diagram, we use several enumerators: Hint, BoardResponse and Rotation. Below, we have given the Z specification of these enumerators.

\begin{axdef}
Rotation == \{0, 90, 180, 270\}
\end{axdef}

\begin{axdef}
BoardResponse == \{FAILED, SUCCESS, WIN\}
\end{axdef}

\begin{axdef}
Hint == \{NORTH, EAST, SOUTH, WEST, NORTH\_EAST, \\ \t1 SOUTH\_EAST, SOUTH\_WEST, NORTH\_WEST\}
\end{axdef}

\section{Classes}
We distinguish between two types of coordinates: absolute coordinates and relative coordinates. The absolute coordinates describe the coordinates of individual tiles on the Board; these coordinates have an x- and y-value, which are both natural numbers. The relative coordinates are used by Robots to make a move request; the relative coordinates also have an x- and y-value, but these numbers are integers. Since the move requests are communicated to the Board, the Board itself will determine the absolute coordinates that belong to these relative coordinates.

\begin{class}{Null}
\upharpoonright (selfRef) \\
\begin{classcom}
This is a class representing nothing.
\end{classcom} \\
\begin{schema}{selfRef}
output! : Null
\where
output! = self
\end{schema}
\end{class}

\begin{class}{AbsoluteCoord}
\upharpoonright (x, y, BoardWith, BoardHeight) \\
\begin{axdef}
BoardWith : \nat
\end{axdef} \\
\begin{axdef}
BoardHeight : \nat
\end{axdef} \\
\begin{state}
x : \nat \cup \{0\} \\
y : \nat \cup \{0\}
\where
x < BoardWidth \\
y < BoardHeight
\end{state}
\end{class}

\begin{class}{RelativeCoord}
\upharpoonright (x, y) \\
\begin{state}
x : \num \\
y : \num
\end{state}
\end{class}

\begin{axdef}
addAbtoRel == (AbsoluteCoord \cross RelativeCoord) \rightarrow (AbsoluteCoord \union Null)
\where
\forall (a,b) : AbsoluteCoord \cross RelativeCoord | \\ \t1
\IF (0 <= a.x + b.x < BoardWidth \\ \t2
0 <= a.x + b.x < BoardHeigth) \\ \t1
\THEN
addAbtoRel(a,b).x = a.x + b.x \\ \t2
addAbtoRel(a,b).y = a.y + b.y \\ \t1
\ELSE addAbtoRel(a,b) = Null.selfRef
\end{axdef}

Tile class has five specializations: NormalTile, HomeTile, HintTile, ConveyerTile and BrokenRobotTile. A Tile has a type, which is either of the above specializations, and a field 'occupier', which describes the robot that currently occupies the tile. Note that a tile does not necessarily have to be occupied by a robot, so occupies can also be null. All the specializations of Tile inherit the characteristics of Tile.
\begin{class}{Tile}
\upharpoonright (type, occupier) \\
\begin{state}
type : \{NormalTile, HomeTile, ConveyorTile, \\ \t1 BrokenRobotTile, HintTile\} \\
occupier : Robot \cup Null
\end{state} \\
\begin{schema}{clear}
\Delta (occupier) \\
\where
occupier' = Null.selfRef
\end{schema} \\
\begin{schema}{put}
\Delta (occupier) \\
input? : Robot
\where
occupier' = input?
\end{schema}
\end{class}

\begin{class}{NormalTile}
Tile
\end{class}

\begin{class}{HintTile}
Tile
\end{class}

\begin{class}{BrokenRobotTile}
Tile
\end{class}

\begin{class}{HomeTile}
Tile \\
\upharpoonright (target) \\
\begin{classcom}
A HomeTile always belongs to one specific Robot, the target robot.
\end{classcom} \\
\begin{state}
target : Robot
\end{state}
\end{class}

\begin{class}{ConveyorTile}
Tile \\
\upharpoonright (rotation) \\
\begin{classcom}
A ConveyerTile has a certain rotation, which influences the direction in which a robot is transported.
\end{classcom} \\
\begin{state}
rotation : Rotation
\end{state}
\end{class}

The Board class maintains several invariants and has a variety of methods, since it is one of the major components of the game.

\begin{class}{Board}
\upharpoonright (tiles, robots) \\
\begin{state}
tiles : \power (AbsoluteCoord \fun Tile) \\
robots : \power (Robot \fun Rotation)
\where
\forall r : robots | \exists c : \dom tiles @  \\ \t1 tiles(c).type = HomeTile \wedge tiles(c).target = r \wedge \\ \t1
\exists d : \dom tiles @ d.occupier = r \wedge Reachable(c, d)
\end{state} \\
\begin{schema}{Occupied}
coord? : AbsoluteCoord \\
output! : \bool
\where
output! = (tiles(coord).type = BrokenRobotTile \; \; \vee \\ \t1
tiles(coord).occupier \not = Null
\end{schema} \\
\begin{schema}{ConveyorUnitDest}
coordA? : AbsoluteCoord \\
coordB? : AbsoluteCoord \\
output! : \bool
\where
output! = \neg Occupied(coordB) \\ \t1
        tiles(input?).type = ConveyorTile \\ \t1
        tiles(input?).rotation = 0 \Rightarrow \\ \t2 input?.x = output!.x + 1 \wedge input?.y = output!.y \\ \t1
        tiles(input?).rotation = 90 \Rightarrow \\ \t2 input?.x = output!.x \wedge input?.y = output!.y - 1 \\ \t1
        tiles(input?).rotation = 180 \Rightarrow \\ \t2 input?.x = output!.x - 1 \wedge input?.y = output!.y \\ \t1
        tiles(input?).rotation = 270 \Rightarrow \\ \t2 input?.x = output!.x \wedge input?.y = output!.y + 1) \\ \t1
\end{schema} \\
\znewpage
\begin{schema}{ConveyorDest}
input? : AbsoluteCoord \\
output! : AbsoluteCoord
\where
\IF ConveyorUnitDest(input?, output!) \\
\THEN output! = input? \\
\ELSE output! = \exists c : AbsoluteCoord | \\ \t2 ConveyorUnitDest(input?, c) \\ \t2 ConveyorDest(c, output!)
\end{schema} \\
\begin{schema}{Adjacent}
coordA? : AbsoluteCoord \\
coordB? : AbsoluteCoord \\
output! : \bool
\where
output! = |\!coordA.x - coordB.x\!| + \\ \t1 |\!coordA.y - coordB.y\!| = 1 \\ \t1
\neg Occupied(coordB)
\end{schema} \\
\begin{schema}{Reachable}
coordA? : AbsoluteCoord \\
coordB? : AbsoluteCoord \\
output! : \bool
\where
output! = ConveyorDest(coordA?) = coordB? \: \vee \\ \t1 (\exists c : AbsoluteCoord | \\ \t2 Adjacent(ConveyorDest(coordA?), c) \: \wedge \\ \t2 ConveyorDest(c) = output!) \: \vee \\ \t1
(\exists c,d : AbsoluteCoord | \\ \t2 Adjacent(ConveyorDest(coordA?), c) \: \wedge \\ \t2 ConveyorDest(c) = d \: \wedge \\ \t2 Reachable(d, output!))
\end{schema} \\
\znewpage
\begin{classcom}
Initialize is called from the outside world, when a new game has to be started. Note that the tiles and robots in this method are read from an input file. After the tiles and robots have been initiated, the Board pre-initializes the Controller and initializes all the robots. Finally, the Board post-initializes the Controller with the initialized Robots and the Board.
\end{classcom} \\
\begin{schema}{Initialize}
\Delta (tiles, robots, controller)
\where
tiles' = createTiles \\
robots' = createRobots \\
controller'.preInitialize \\
\forall r : \dom Robots | r.initialize(controller)
\end{schema} \\
\begin{classcom}
The method below creates a set of robots from a file. This is not further specified.
\end{classcom} \\
\begin{schema}{createRobots}
output! : \power (Robot \fun Rotation)
\end{schema} \\
\begin{classcom}
The method below creates a set of tiles from a file. This is not further specified.
\end{classcom} \\
\begin{schema}{createTiles}
output! : \power (AbsoluteCoord \fun Tile)
\end{schema} \\
\begin{classcom}
CanReset is used by the Controller to let the Board know that the game has ended and that the Board can reset. The Board can reset whenever there is HomeTile on the Board that is occupied by the target robot.
\end{classcom} \\
\begin{schema}{canReset}
\where
(\exists t : Tile | t.type = HomeTile \\ \t1
t.occupier = t.target) \Rightarrow board.reset
\end{schema} \\
\begin{classcom}
When the Board resets itself, the tiles, the controller and the robots are deleted.
\end{classcom} \\
\begin{schema}{reset}
\Delta (tiles, robots)
\where
tiles' = Null.selfRef \\
robots' = Null.selfRef \\
controller' = Null.selfRef
\end{schema} \\
\znewpage
\begin{classcom}
We make a snapshot by copying all the tiles and adding a mapping (same as for the original board). Next, with this snapshot the view can display the board.
\end{classcom} \\
\begin{schema}{requestSnapShot}
boardSnapshot : BoardSnapShot \\
output! : BoardSnapshot
\where
\forall (c,t) : tiles |  \\ \t1
\exists (d,u) : boardSnapshot | c = d \: \wedge \\ \t2 t \not = u \wedge t.type = u.type \: \wedge \\ \t2 t.occupier = u.occupier \: \wedge \\ \t2 t.type = HomeTile \Rightarrow t.target = u.target \: \wedge \\ \t2 t.type = ConveyorTile \Rightarrow t.rotation = u.rotation \\
output! = boardSnapshot
\end{schema} \\
\begin{classcom}
GetHint is used by the Board to generate a hint. It takes the robot that requested a move as input-parameter. It checks where the HomeTile of the robot is and according to that gives the appropiate hint.
\end{classcom} \\
\begin{schema}{getHint}
robot? : Robot \\
output!: Hint
\where
\exists t, t1: \ran tiles | t.occupier = robot? @\\ \t1
t1.type = HomeTile @ t1.target = robot? \\ \t1
\exists coord, coord1: AbsoluteCoord | tiles(coord) = t \\ \t2
\IF coord.x = coord1.x \\ \t2
\THEN \\ \t3
\IF coord.y < coord1.y \\ \t3
\THEN output! = SOUTH \\ \t3
\ELSE output! = NORTH \\ \t2
\ELSE \\ \t3
\IF coord.x > coord1.x \\ \t3
\THEN output! = getHintLeft(coord , coord1) \\ \t3
\ELSE output! = getHintRight(coord, coord1)
\end{schema} \\
\znewpage
\begin{classcom}
This method is a sub method of getHint, used to deal with HomeTile to the left of the robot.
\end{classcom} \\
\begin{schema}{getHintLeft}
absCoord? : AbsoluteCoord \\
absCoord1? : AbsoluteCoord \\
output! : Hint
\where
\IF absCoord?.y = absCoord1?.y \\
\THEN output! = WEST \\
\ELSE \\ \t1
\IF absCoord?.y < absCoord1.y \\ \t1
\THEN \exists hint: \{SOUTH, WEST, SOUTH\_WEST\} | \\ \t3 output! = hint \\ \t1
\ELSE \exists hint: \{NORTH, WEST, WEST\_NORTH\} | \\ \t3 output! = hint
\end{schema}\\
\begin{classcom}
This method is a sub method of getHint, used to deal with HomeTile to the right of the robot.
\end{classcom} \\
\begin{schema}{getHintRight}
absCoord? : AbsoluteCoord \\
absCoord1? : AbsoluteCoord \\
output! : Hint
\where
\IF absCoord?.y = absCoord1?.y \\
\THEN output! = EAST \\
\ELSE \\ \t1
\IF absCoord?.y < absCoord1.y \\ \t1
\THEN \exists hint: \{SOUTH, EAST, EAST\_SOUTH\} | \\ \t3 output! = hint \\ \t1
\ELSE \exists hint: \{NORTH, EAST, NORTH\_EAST\} | \\ \t3 output! = hint
\end{schema} \\
\begin{classcom}
getValidTiles corresponds to the private method in Board. The output is a pair of two valid tiles. A tile is valid for an exchange if it is not a HomeTile or a HintTile. Off course, the invariant also holds for this function.
\end{classcom} \\
\begin{schema}{getValidTiles}
output! : (Tile \cross Tile)
\where
\exists t,t1: Tile | t \not = t1  @ \\ \t1
t.type \not = HomeTile \\ \t1
t.type \not = HintTile \\ \t1
t1.type \not = HomeTile \\ \t1
t1.type \not = HintTile \implies \\ \t1
output! = (t \cross t1)
\end{schema}
\znewpage
\begin{classcom}
Method to exchange the positions of two tiles on the board. These tiles should be valid. If a robot is on one of the tiles, it moves along.
\end{classcom} \\
\begin{schema}{requestTilesExchange}
\Delta (tiles)
output! : ({Robot \union Null} \cross {Robot \union Null}) \\
(tile, tile1) : Tile \cross Tile
\where
(tile, tile1)  = getValidTile \\
\exists abCoord, abCoord1 : AbsoluteCoord | tiles(abCoord) = tile \\ \t1
tiles(abCoord1) = tile1 \implies \\ \t2
tiles(abCoord)' = tile1 \\ \t2
tiles(abCoord1)' = tile \\ \t2
\exists n : notifyView \\ \t2
\exists r : Robot | tile.occupier = r \implies \\ \t3
\exists rotation : Rotation | robots(r)' = rotation \\ \t3
\exists n : notifyStateChange | n.input? = r \\ \t2
\exists r1 : Robot | tile1.occupier = r \implies \\ \t3
\exists rotation1 : Rotation | robots(r1)' = rotation1 \\ \t3
\exists n1: notifyStateChange | n1.input? = r1 \\ \t2
\IF tile.occupier = Null.selfRef \\ \t3
tile1.occupier \not = Null.selfRef\\ \t2
\THEN \exists m : moveConveyorSwitch | \\ \t3 \t1 m.absoluteCoord? = abCoord1 \\ \t3
\IF tile1.occupier = Null.selfRef \\ \t3
tile.occupier \not = Null.selfRef \\ \t2
\THEN \exists m : moveConveyorSwitch | \\ \t3 \t1 m.absoluteCoord? = abCoord1
\end{schema} \\
\begin{classcom}
Method to deal with the fact that a robot is part of a tile exchange.
\end{classcom} \\
\begin{schema}{specialExchange}
\Delta (robots)
tile? : Tile
\where
\exists r : Robot | tile.occupier = r \implies \\ \t1
controller.notifyAutoMovement(r) \\ \t1
\exists rotation : Rotation | robots(r)' = rotation
\end{schema} \\
\begin{classcom}
Make sure that a robot on a conveyor belt is moved to the proper tile.
\end{classcom} \\
\begin{schema}{moveConveyorSwitch}
absoluteCoord? : AbsoluteCoord \\
\where
moveConveyorSwitchSub(absoluteCoord?, 0,1) \\
moveConveyorSwitchSub(absoluteCoord?, 1,0) \\
moveConveyorSwitchSub(absoluteCoord?, 0,-1) \\
moveConveyorSwitchSub(absoluteCoord?, -1,0) \\
\end{schema}
\znewpage
\begin{classcom}
Subfunction of moveConveyorSwitch, does the actual calculation of the placement of the robot.
\end{classcom} \\
\begin{schema}{moveConveyorSwitchSub}
absoluteCoord? : AbsoluteCoord \\
x? : int \\
y? : int \\
relCoord : RelativeCoord
\where
relCoord.x = x? \\
relCoord.y = y? \\
\IF tiles(addAbtoRel(absoluteCoord?, relCoord)).type = ConveyorTile \\ \t1
tiles(absoluteCoord?).occupier = Null \\ \t1
\exists r : Robot | \\ \t2 tiles(addAbtoRel(absoluteCoord?, relCoord)).occupier = r \\
\THEN moveConveyorSwitch(addAbtoRel(absoluteCoord?, relCoord)\\ \t1
saveLocation(addAbtoRel(absoluteCoord?, relCoord), r)
\end{schema} \\
\begin{classcom}
This functions deals with the moveRequest of the controller
\end{classcom} \\
\begin{schema}{moveRequest}
localCoords? : RelativeCoord \\
robot? : Robot \\
rotation? : Rotation \\
absoluteCoord : AbsoluteCoord \\
output! : BoardResponse
\where
\IF rotation? \not = 0 \\
\THEN response! = moveRotate(localCoords?, robot?, rotation?) \\
\ELSE response! = moveWalk(localCoords?, robot?)
\end{schema} \\
\begin{classcom}
Used to deal with rotations a robot wants to make.
\end{classcom} \\
\begin{schema}{moveRotate}
localCoords? : RelativeCoord \\
robot? : Robot \\
rotation? : Rotation \\
output! : BoardResponse
\where
\IF localCoords.x = 0 \\ \t1
localCoords.y = 0 \\ \t1
 \exists r : robot?.rule.possibleRotations | r = rotation?\\
\THEN output! = SUCCESS \\ \t1
robots(robot?) = (robots(robot?) + rotation?) \mod 360 \\
\ELSE output! = FAILED
\end{schema} \\
\znewpage
\begin{schema}{moveWalk}
localCoords? : RelativeCoord \\
robot? : Robot \\
output! : BoardResponse
\where
\IF robot?.rule.possibleMoves(localCoords?, robots(robot?)) = \\ \t1
possiblePath(robot?) \wedge rotation? = 0 \\
\THEN absoluteCoord = calculateNewLocation(localCoords?, robot?)\\ \t1
\IF absoluteCoord = Null \\ \t1
\THEN output! = FAILED \\ \t1
\ELSE output! = checkTile(absoluteCoord?, robot?)\\ \t2
\ELSE output! = FAILED
\end{schema} \\
\begin{schema}{checkTile}
absoluteCoord? : AbsoluteCoord \\
robot? : Robot \\
output! : BoardResponse
\where
saveLocation(absoluteCoord?, robot?) \\
\IF tiles(absoluteCoord?).type = HomeTile \\ \t1
tiles(absoluteCoord?).target = robot? \\
\THEN output! = WIN \\ \
\ELSE  \\ \t1
\IF tiles(absoluteCoord?).type = ConveyorTile \\ \t1
\THEN controller.notifyAutomovement(robot?) \\ \t1
\IF tiles(absoluteCoord?).type = HintTile \\ \t1
\THEN controller.notifyHint(Board.getHint(robot?), robot?) \\ \t1
output! = SUCCESS
\end{schema} \\
\begin{schema}{saveLocation}
absCoords? : AbsoluteCoord \\
robot? : Robot
\where
\exists t : Tile | t.occupier = robot? \wedge t.clear\\
tiles(absCoords?).put(robot) \\
viewer.notifyView
\end{schema}
\znewpage
\begin{schema}{calculateNewLocation}
localCoords? : RelativeCoord \\
robot? : Robot \\
absoluteCoord : AbsoluteCoord \\
output! : (AbsoluteCoord \union Null)
\where
\IF checkPath(localCoords?, robot?) \\
\THEN absoluteCoord = firstConveyor(c, path) \\ \t1
\exists s : ConveyorDest | s.input? = absoluteCoord @ \\ \t2
output! = s.output! \\
\ELSE output! = Null.selfRef
\end{schema} \\
\begin{schema}{checkPath}
localCoords? : RelativeCoord \\
robot? : Robot \\
path : seq RelativeCoord \\
absoluteCoord : AbsoluteCoord \\
bool : \bool \\
output! : \bool
\where
bool = false \\
path = robot?.rule.possibleMoves(localCoords? \cross robots(robot?)) \\
\exists (c, t) : tiles | t.occupier = robot? \implies \\ \t1
\forall (int, coord) : path | \\ \t2
((addAbtoRel(c, coord) = Null \vee \\ \t2
(tiles(addAbtoRel(c, coord).type = BrokenRobotTile) \vee \\ \t2
\exists r: Robot | tiles(addAbtoRel(c, coord).occupier = r )\implies \\ \t3
\exists (int1, coord1) : path | int1 < int \\ \t3
tiles(addAbtoRel(c, coord1).type = ConveyorTile \\ \t3
absoluteCoord = addAbtoRel(c, coord1)) \implies \\ \t4
bool = \true \\ \t4
output! = bool
\end{schema} \\
\begin{schema}{firstConveyor}
firstConveyor == (AbsoluteCoord \cross seq RelativeCoord) \fun \\ \t1 AbsoluteCoord
\where
\exists absoluteCoord : AbsoluteCoord |  \\ \t1
\exists (int, coord): seq RelativeCoord |  \\ \t2
(tiles(addAbtoRel(absoluteCoord, coord).type = conveyorTile \\ \t2
conveyorInPath(absoluteCoord, (int, coord)) = \\ \t3 addAbtoRel(absoluteCoord, coord)) \implies \\ \t3
\forall (int1, coord1): seq RelativeCoord | \\ \t3
\IF tiles(addAbtoRel(absoluteCoord, coord1).type = conveyorTile \\ \t3
\THEN int <= int1
\end{schema}
\end{class}

A BoardSnapshot is simply a copy of the current state of the board; therefore, the BoardSnapshot maintains a mapping of absolute coordinates to the tiles.
\begin{class}{BoardSnapshot}
\upharpoonright (tiles) \\
\begin{state}
tiles : \power (AbsoluteCoord \fun Tile) \\
\end{state}
\end{class}

A Rule consists of a list of possible moves and a list of possible rotations. The possible moves are described in terms of local coordinates, since a robot does not know its exact location on the board. Since each robot has a certain rotation, possibleMoves maps relative coordinates and rotations to each possible relative coordinate that the robot can move to. The list of possible rotations is simply a list of all rotations.
\begin{class}{Rule}
\upharpoonright (possibleMoves, possibleRotations) \\
\begin{state}
possibleMoves : \power ((RelativeCoord \times Rotation) \psurj \\ \t1 \seq RelativeCoord) \\
possibleRotations : \power Rotation
\end{state} \\
\end{class}

A Robot has knowledge of the Controller and maintains a rule-attribute, describing the ruleset of the Robot.
\begin{class}{Robot}
\upharpoonright (rules) \\
\begin{state}
rules : Rules \\
controller : Controller
hint : Hints
\end{state}\\
\begin{schema}{initialize}
\Delta (controller, rules, hint) \\
controller? : Controller \\
rules? : Rules
\where
controller' = controller? \\
rules' = rules? \\
hint' = Null.selfRef
\end{schema}\\
\begin{schema}{notifyAutoMovement}
\Delta (hint)
\where
hint' = Null.selfRef
\end{schema}\\
\begin{schema}{notifyHint}
\Delta (hint) \\
hint? : Hint
\where
hint' = \{hint?\} \union hint
\end{schema}
\end{class}

The Viewer has no knowledge of the Board; every change of the Board must be communicated to the Viewer via the Controller. The variable 'boardChanged' is used as a flag to indicate that the board has changed and the Viewer has not yet updated the view.
\begin{class}{Viewer}
\begin{state}
controller : Controller \\
boardChanged : \bool
\end{state}\\
\begin{schema}{initialize}
\Delta (controller)
\where
\exists a : addViewer | a.viewer? = self \implies \\ \t1
a.output! = controller' \\ \t1
boardChanged' = \false
\end{schema}\\
\begin{schema}{notifyGameOver}
robot? : Robot
\where
Fireworks
\end{schema}\\
\begin{schema}{notifyStateChange}
\Delta (boardChanged)
\where
boardChanged' = \true
\end{schema}\\
\begin{schema}{updateView}
\Delta (controller, snapShot)
\where
\IF boardChanged = \true \\
\THEN \exists r: requestBoardStatus | output! = snapShot' \\ \t1
boardChanged' = \false \\ \t1
controller' = controller \\
\ELSE snapShot' = snapShot \\ \t1
boardChanged' = \false \\ \t1
controller' = controller
\end{schema}
\end{class}

The controller has knowledge about the Board and the Viewer; it also maintains a list of the robots.
\begin{class}{Controller}
\begin{state}
board : Board \\
robots : \power Robot
viewer : Viewer
\end{state}\\
\begin{schema}{addViewer}
\Delta (viewer) \\
viewer? : Viewer \\
output! : Controller
\where
\IF viewer = Null \\
\THEN viewer' = viewer? \\ \t1
output! = self \\
\ELSE output! = Null.selfRef
\end{schema}\\
\begin{schema}{notifyAutoMovement}
robot? : Robot
\where
robot?.notifyAutomovement
\end{schema}\\
\begin{schema}{moveRequest}
localCoords? : RelativeCoord \\
robot? : Robot \\
rotation? : Rotation \\
ouput! : BoardResponse
\where
\exists m : Board.moveRequest | m.localCoords? = localCoords \\ \t1
m.robot? = robot? \\ \t1
m.rotation? = rotation? \\ \t1
m.output! = output! \\ \t1
\IF output! = WIN \\ \t1
\THEN \forall r : Robot | r.terminate \\ \t2
\exists n : viewer.notifyGameOver | n.robot? = robot? \\ \t2
board.canReset \\ \t2
terminate
\end{schema}\\
\begin{schema}{notifyHint}
hint? : Hint \\
robot? : Robot
\where
\exists n : robot.notifyHint | n.hint? = hint?
\end{schema}\\
\begin{schema}{notifyView}
\where
\exists n : notifyStateChange
\end{schema}
\znewpage
\begin{schema}{preInitialize}
\Delta (board, robots, vieuwer)
\where
board' = Null.selfRef \\
robots' = Null.selfRef \\
viewer' = Null.selfRef
\end{schema}\\
\begin{schema}{postInitialize}
\Delta (board, robots) \\
board? : Board \\
robots? : \power Robot
\where
board' = board \\
robots' = robots?
\end{schema}\\
\begin{schema}{removeViewer}
\Delta (viewer, robots, board)
\where
viewer' = Null.selfRef \\
robots' = robots \\
board' = board
\end{schema}\\
\begin{schema}{requestBoardStatus}
output! : BoardSnapShot
\where
\exists r : requestSnapShot | r.output! = output!
\end{schema}
\end{class}

\end{document}

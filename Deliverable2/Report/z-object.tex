\documentclass[12pt]{article}
\usepackage{ltcadiz}
\usepackage{listings}
\lstset{breaklines=true, numbers=left}

\title{Report on OGO 2.2 Softwarespecification\\ Assignment 1a}
\author{
        Femke Jansen and Lasse Blaauwbroek OGO 2.2 group 6 \\
                Department of Computer Science\\
        Technical University Eindhoven\\
}
\date{\today}

\begin{document}

\maketitle

\begin{abstract}

\end{abstract}

\section{Basic Axiomatic definitions}
We define several attributes that we will use in later definitions in our Z schemas. Since some returned objects in our class diagram may contain null values, we have defined it as a separate set. \\
In our class diagram, we use several enumerators: Hint, BoardResponse and Rotation. Below, we have given the Z specification of these enumerators.
\begin{class}{Null}
\begin{classcom}
This is a class representing nothing.
\end{classcom}
\end{class}

\begin{axdef}
Rotation == \{0, 90, 180, 270\}
\end{axdef}

\begin{axdef}
BoardResponse == \{FAILED, SUCCESS, WIN\}
\end{axdef}

\begin{axdef}
Hint == \{NORTH, EAST, SOUTH, WEST, NORTH\_EAST, \\ \t1 EAST\_SOUTH, SOUTH\_WEST, WEST\_NORTH\}
\end{axdef}

\section{Classes}
We distinguish between two types of coordinates: absolute coordinates and relative coordinates. The absolute coordinates describe the coordinates of individual tiles on the Board; these coordinates have an x- and y-value, which are both natural numbers. The relative coordinates are used by Robots to make a move request; the relative coordinates also have an x- and y-value, but these numbers are integers. Since the move requests are communicated to the Board, the Board itself will determine the absolute coordinates that belong to these relative coordinates.

\begin{class}{AbsoluteCoord}
\upharpoonright (x, y, BoardWith, BoardHeight) \\
\begin{axdef}
BoardWith : \nat
\end{axdef} \\
\begin{axdef}
BoardHeight : \nat
\end{axdef} \\
\begin{state}
x : \nat \cup \{0\} \\
y : \nat \cup \{0\}
\where
x < BoardWidth \\
y < BoardHeight
\end{state}
\end{class}

\begin{class}{RelativeCoord}
\upharpoonright (x, y) \\
\begin{state}
x : \num \\
y : \num
\end{state}
\end{class}

Tile class has five specializations: NormalTile, HomeTile, HintTile, ConveyerTile and BrokenRobotTile. A Tile has a type, which is either of the above specializations, and a field 'occupier', which describes the robot that currently occupies the tile. Note that a tile does not necessarily have to be occupied by a robot, so occupies can also be null. All the specializations of Tile inherit the characteristics of Tile.
\begin{class}{Tile}
\upharpoonright (type, occupier) \\
\begin{state}
type : \{NormalTile, HomeTile, ConveyorTile, \\ \t1 BrokenRobotTile, HintTile\} \\
occupier : Robot \cup Null
\end{state}
\end{class}

\begin{class}{NormalTile}
Tile
\end{class}

\begin{class}{HintTile}
Tile
\end{class}

\begin{class}{BrokenRobotTile}
Tile
\end{class}

\begin{class}{HomeTile}
Tile \\
\upharpoonright (target) \\
\begin{classcom}
A HomeTile always belongs to one specific Robot, the target robot.
\end{classcom} \\
\begin{state}
target : Robot
\end{state}
\end{class}

\begin{class}{ConveyorTile}
Tile \\
\upharpoonright (rotation) \\
\begin{classcom}
A ConveyerTile has a certain rotation, which influences the direction in which a robot is transported.
\end{classcom} \\
\begin{state}
rotation : Rotation
\end{state}
\end{class}

The Board class maintains several invariants and has a variety of methods, since it is one of the major components of the game.

\begin{class}{Board}
\upharpoonright (tiles, robots) \\
\begin{state}
tiles : \power (AbsoluteCoord \fun Tile) \\
robots : \power (Robot \fun Rotation)
\where
\forall r : robots | \exists c : \dom tiles @  \\ \t1 tiles(c).type = HomeTile \wedge tiles(c).target = r \wedge \\ \t1
\exists d : \dom tiles @ d.occupier = r \wedge Reachable((c, d))
\end{state} \\
\begin{schema}{Occupied}
coord? : AbsoluteCoord \\
output! : \bool
\where
output! = (tiles(coord).type = BrokenRobotTile \; \; \vee \\ \t1
tiles(coord).occupier \not = Null
\end{schema} \\
\begin{schema}{ConveyorUnitDest}
coordA? : AbsoluteCoord \\
coordB? : AbsoluteCoord \\
output! : \bool
\where
output! = \neg Occupied(coordB) \\ \t1
        tiles(input?).type = ConveyorTile \\ \t1
        tiles(input?).rotation = 0 \Rightarrow \\ \t2 input?.x = output!.x + 1 \wedge input?.y = output!.y \\ \t1
        tiles(input?).rotation = 90 \Rightarrow \\ \t2 input?.x = output!.x \wedge input?.y = output!.y - 1 \\ \t1
        tiles(input?).rotation = 180 \Rightarrow \\ \t2 input?.x = output!.x - 1 \wedge input?.y = output!.y \\ \t1
        tiles(input?).rotation = 270 \Rightarrow \\ \t2 input?.x = output!.x \wedge input?.y = output!.y + 1) \\ \t1
\end{schema} \\
\znewpage
\begin{schema}{ConveyorDest}
input? : AbsoluteCoord \\
output! : AbsoluteCoord
\where
\IF ConveyorUnitDest(input?, output!) \\
\THEN output! = input? \\
\ELSE output! = \exists c : AbsoluteCoord | \\ \t2 ConveyorUnitDest(input?, c) \\ \t2 ConveyorDest(c, output!)
\end{schema} \\
\begin{schema}{Adjacent}
coordA? : AbsoluteCoord \\
coordB? : AbsoluteCoord \\
output! : \bool
\where
output! = |\!coordA.x - coordB.x\!| + \\ \t1 |\!coordA.y - coordB.y\!| = 1 \\ \t1
\neg Occupied(coordB)
\end{schema} \\
\begin{schema}{Reachable}
coordA? : AbsoluteCoord \\
coordB? : AbsoluteCoord \\
output! : \bool
\where
output! = ConveyorDest(coordA?) = coordB? \: \vee \\ \t1 (\exists c : AbsoluteCoord | \\ \t2 Adjacent(ConveyorDest(coordA?), c) \: \wedge \\ \t2 ConveyorDest(c) = output!) \: \vee \\ \t1
(\exists c,d : AbsoluteCoord | \\ \t2 Adjacent(ConveyorDest(coordA?), c) \: \wedge \\ \t2 ConveyorDest(c) = d \: \wedge \\ \t2 Reachable(d, output!))
\end{schema}
\end{class}

A BoardSnapshot is simply a copy of the current state of the board; therefore, the BoardSnapshot maintains a mapping of absolute coordinates to the tiles.
\begin{class}{BoardSnapshot}
\upharpoonright (tiles) \\
\begin{state}
tiles : \power (AbsoluteCoord \fun Tile) \\
\end{state}
\end{class}

A Rule consists of a list of possible moves and a list of possible rotations. The possible moves are described in terms of local coordinates, since a robot does not know its exact location on the board. Since each robot has a certain rotation, possibleMoves maps relative coordinates and rotations to each possible relative coordinate that the robot can move to. The list of possible rotations is simply a list of all rotations.
\begin{class}{Rule}
\upharpoonright (possibleMoves, possibleRotations) \\
\begin{state}
possibleMoves : \power ((RelativeCoord \times Rotation) \psurj \\ \t1 \seq RelativeCoord) \\
possibleRotations : \power Rotation
\end{state}
\end{class}

A Robot has knowledge of the Controller and maintains a rule-attribute, describing the ruleset of the Robot.
\begin{class}{Robot}
\upharpoonright (rules) \\
\begin{state}
rules : Rules \\
controller : Controller
\end{state}
\end{class}

The Viewer has no knowledge of the Board; every change of the Board must be communicated to the Viewer via the Controller. The variable 'boardChanged' is used as a flag to indicate that the board has changed and the Viewer has not yet updated the view.
\begin{class}{Viewer}
\begin{state}
controller : Controller \\
boardChanged : \bool
\end{state}
\end{class}

The controller has knowledge about the Board and the Viewer; it also maintains a list of the robots.
\begin{class}{Controller}
\begin{state}
board : Board \\
robots : \power Robot
viewer : Viewer
\end{state}
\end{class}

\end{document} 
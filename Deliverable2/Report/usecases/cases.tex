\begin{alltt}
\rm
The next use case illustrates the initialize state in the High Level Message Sequence Chart. In this use case, all of the different components (except for the viewer) will get initialized.

Initialize

- Pre-conditions: True
- Trigger: The user presses the start game button.
- Guarantee: A new game is initiated.
- Main Scenarios
    (a) Initialize board
    (b) The board pre-initializes the controller
    (c) The board initializes all robots (one by one)
    (d) The board post-initializes the controller with all the robots
- Alternatives

The "End game" use case illustrates the End game state in the High Level Message Sequence Chart. This use case deals with ending the game once a robot has reached its home tile.

End game
- Pre-conditions: The game is initiated.
- Trigger:  A robot has reached its home tile.
- Guarantee: The game is properly terminated.
- Main Scenarios
    (a) The controller sends a terminate message to the every robot (except for the one that has won), hereby notifying them that they have lost and making them terminate
    (b) Notify the viewer that the game is over and which robot has won
    (c) Viewer displays fireworks
    (d) Remove viewer from controller
    (e) All remaining objects (except for the board) can now terminate, the board resets
- Alternatives

The "Tiles exchange" use case illustrates both the Special exchange and the Ordinary exchange state in the High Level Message Sequence Chart. In the "normal" case, two tiles are swapped (no robots on it and they are normal tiles). In the special cases, either a robot is on a tile and needs to be notified that its position has changed and he has been rotated. Or a conveyor belt is switched and rotated (or a combination of both).

Tiles exchange
- Pre-conditions: The game is initiated.
- Trigger:  A robot has moved.
- Guarantee: Two tiles are exchanged
- Main Scenarios
    (a) Two valid tiles are chosen
    (b) The chosen tiles are exchanged (ordinary exchange)
    (c) The board makes a snapshot and sends this to the controller
    (d) The controller notifies the viewer that there are changes and sends the snapshot along with this
- Alternatives
    (b.1) If a chosen tile contains a robot, the robot will rotate (this can be the same) (special exchange)
    (c.1) The board notifies the controller that a robot has moved
    (d.1) The controller notifies the robot that is has been moved
    (e.1) The board makes a snapshot and sends this to the controller
    (f.1) The controller notifies the viewer that there are changes and sends the snapshot along with this
    
    (b.2) If a chosen tile is a conveyor tile, the rotation of the conveyor tile might change (can also be the same) (special exchange)
    (c.2) The board makes a snapshot and sends this to the controller
    (d.2) The controller notifies the viewer that there are changes and sends the snapshot along with this

The "Robot move request" use case illustrates the Robot move request state in the High Level Message Sequence Chart. In this use case, the robot request the controller if it can make a move. The controller forwards this request to the board, which then decides. Note that the board here checks if the move is allowed. This also deals with the fact that, if a robot wants to move to its home tile and this is occupied, the robot still may move (and moves) to this tile.

Robot move Request
- Pre-conditions: The game is initiated. And the indicated position
    is a empty tile.
- Trigger: A robot requests to move his position.
- Guarantee: The robot is moved to the indicated position.
- Main Scenarios
    (a) The robot requests the controller if it may move
    (b) The controller sends the request to the board
    (c) The board checks the request
- Alternatives

The "Return Normal tile" use case illustrates the Return Normal tile state in the High Level Message Sequence Chart. This use case always comes after the Robot move request use case. This use case handles with a robot for which his move request to go to a normal tile has been approved.

Return Normal tile
- Pre-conditions: The game is initiated. And the indicated position
    is a normal tile, move request approved.
- Trigger: A robot requests to move his position to a normal tile.
- Guarantee: The robot is moved to the indicated position which
    is a normal tile.
- Main Scenarios
    (a) The robot is moved to the indicated position which is a normal tile
    (b) The board notifies the controller that the move request has been approved
    (c) The controller notifies the robot that his move request has been approved and that the robot has been moved to the requested position
    (d) The board makes a snapshot and sends this to the controller
    (e) The controller notifies the viewer that there are changes and sends the snapshot along with this
- Alternatives

The "Return Hint tile" use case illustrates the Return Hint tile state in the High Level Message Sequence Chart. This use case always comes after the Robot move request use case. This use case handles with a robot for which his move request to go to a hint tile has been approved.

Return Hint tile
- Pre-conditions: The game is initiated. The indicated position
    is a hint tile, move request approved.
- Trigger: A robot requests to move his position to a hint tile.
- Guarantee: The robot is moved to the indicated position which
    is a hint tile and a hint is given to the player.
- Main Scenarios
    (a) The robot is moved to the indicated position which is a hint tile
    (b) The board notifies the controller that the move request has been approved
    (c) The controller notifies the robot that his move request has been approved and that the robot has been moved to the requested position
    (d) The board makes a snapshot and sends this to the controller
    (e) The controller notifies the viewer that there are changes and sends the snapshot along with this
    (f) The board generates a hint for the robot
    (g) The board notifies the controller of the hint and the robot it needs to be send to
    (h) The controller sends the hint to the robot
- Alternatives

The "Return Conveyor tile" use case illustrates the Return Conveyor tile state in the High Level Message Sequence Chart. This use case always comes after the Robot move request use case. This use case handles with a robot for which his move request to go to a conveyor tile has been approved.

Return Conveyor tile
- Pre-conditions: The game is initiated. And the indicated position
    is a conveyor tile, move request approved.
- Trigger: A robot requests to move his position to a conveyor
    belt tile.
- Guarantee: The robot is moved to the indicated position which
    is a conveyor belt tile.
- Main Scenarios
    (a) The robot is moved to the indicated position which is a conveyor
    (b) The board notifies the controller that the move request has been approved
    (c) The controller notifies the robot that his move request has been approved and that the robot has been moved to the requested position
    (d) The board calculates the new position (after movement over the conveyor tile)
    (e) The board makes a snapshot and sends this to the controller
    (f) The controller notifies the viewer that there are changes and sends the snapshot along
    (g) The board notifies the controller that a robot has been moved automatically
    (h) The controller notifies the robot that it has been moved automatically
- Alternatives

The "Return Home tile" use case illustrates the Return Home tile state in the High Level Message Sequence Chart. This use case always comes after the Robot move request use case. This use case handles with a robot for which his move request to go to his home tile has been approved.

Return Home tile
- Pre-conditions: The game is initiated. And the indicated position
    is its home tile, move request approved.
- Trigger: A robot requests to move his position to a home tile.
- Guarantee: The robot is moved to the indicated position which
    is its home tile.
- Main Scenarios
    (a) The robot is moved to the indicated position which is the robots home tile
    (b) The board notifies the controller that the move request has been approved and the robot has won
    (c) The controller notifies the robot that his move request has been approved, that the robot has been moved to the requested position and that the robot has won
    (d) The board makes a snapshot and sends this to the controller
    (e) The controller notifies the viewer that there are changes and sends the snapshot along
- Alternatives

The "Reject move" use case illustrates the Reject move state in the High Level Message Sequence Chart. This use case always comes after the Robot move request use case. This use case handles with a robot for which his move request has been rejected.

Reject move
- Pre-conditions: The game is initiated. A move request of a robot is rejected.
- Trigger: A robot requests to move
- Guarantee: The robot remains on the same position
- Main Scenarios
    (a) The board sends the reject message to the controller
    (b) The controller sends the reject message to the robot
- Alternatives

The "Notify robots" use case illustrates the Notify robots state in the High Level Message Sequence Chart. This use case should deal with robots that have been moved by the board, due to movement of another robot (e.g. a robot blocks the end of the conveyor tile you are on, moves away and you will be pushed of the conveyor tile).

Notify robots
- Pre-conditions: The game is initiated.
- Trigger: A robot is moved due to movement of another robot (by conveyor tile)
- Guarantee: The robot gets a message that is has been moved
- Main Scenarios
    (a) The board sends a message to the controller, notifying it that a robot has been moved
    (b) The controller notifies the robot that it has been moved
- Alternatives

The "Initialize viewer" use case illustrates the Initialize viewer state in the High Level Message Sequence Chart. This use case is used to illustrate how the viewer will be initialized.

Initialize viewer
- Pre-conditions: True
- Trigger: World messages the viewer
- Guarantee: Initialized viewer
- Main Scenarios
    (a) Viewer initializes
    (b) Viewer sends a message to the controller that he is the viewer
- Alternatives

The "Update viewer" use case illustrates the Update viewer state in the High Level Message Sequence Chart. Use case for updating the viewer after something on the board has changed.

Update viewer
- Pre-conditions: The game is initiated
- Trigger: Notify and new snapshot by the controller
- Guarantee: Update viewer
- Main Scenarios
    (a) Update view by the received snapshot
- Alternatives


\end{alltt}

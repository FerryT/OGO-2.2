\documentclass[a4paper,11pt]{article}
\usepackage{a4wide}
\usepackage{eurosym}
\usepackage{graphicx}
\usepackage[english]{babel}
\usepackage{ltcadiz}
\begin{document}
\title{Z schemas (methods)}
\author{J. Peters, s102231}
\maketitle


BEGIN BOARD
Gegevens worden ingelezen vanuit file
\begin{schema}{initialize}
\Delta Board
\where
tiles' = tiles \\
robots' = robots \\
controller.preInitialize \\
\forall r : Robot | r \in \dom(Robots) \implies \\ \t1
r.initialize(controller) \\ 
controller.postInitialize(self, robots')
\end{schema}

\begin{schema}{canReset}
\Delta Board
\where
\IF \exists t : Tile | t.type = HomeTile \\ \t1
t.occupier = t.target \\
\THEN board.reset
\end{schema}

\begin{schema}{reset}
\Delta Board
\where
tiles' = Null \\
robots' = Null
\end{schema}

\begin{schema}{requestSnapShot}
\Xi Board \\
boardSnapshot : BoardSnapShot \\
output! : BoardSnapshot
\where
\forall t : Tile |  \\ \t1
\exists abCoord : AbsoluteCoord | tiles(abCoord) = t \implies \\ \t2
\exists c : cloneTile | c.tile? = t \\ \t3
boardSnapshot(abCoord) = c.output! \\
output! = boardSnapshot
\end{schema}

\begin{schema}{cloneTile}
\Delta Tile \\
tile? : Tile \\
output! : Tile 
\where
tile? \not = output! \\
tile?.type = output!.type \\
tile?.occupier = output!.occupier 
\end{schema}


\begin{schema}{getHint}
\Xi Board \\
robot? : Robot \\
output!: Hint
\where
\exists t, t1: Tile | t.occupier = robot? @\\ \t1
t1.type = HomeTile @ t1.target = robot? \\ \t1
\exists coord, coord1: AbsoluteCoord | tiles(coord) = t \\ \t2
\IF coord.x = coord1.x \\ \t2
\THEN \\ \t3
\IF coord.y < coord1.y \\ \t3
\THEN output! = SOUTH \\ \t3
\ELSE output! = NORTH \\ \t2
\ELSE \\ \t3
\IF coord.x > coord1.x \\ \t3
\THEN output! = getHintLeft(coord , coord1) \\ \t3
\ELSE output! = getHintRight(coord, coord1)
\end{schema}

\begin{schema}{getHintLeft}
\Xi Board \\
absCoord? : AbsoluteCoord \\
absCoord1? : AbsoluteCoord \\
output! : Hint 
\where
\IF absCoord?.y = absCoord1?.y \\
\THEN output! = WEST \\
\ELSE \\ \t1
\IF absCoord?.y < absCoord1.y \\ \t1
\THEN \exists hint: \{SOUTH, WEST, SOUTH\_WEST\} | output! = hint \\ \t1
\ELSE \exists hint: \{NORTH, WEST, NORTH\_WEST\} | output! = hint
\end{schema}

\begin{schema}{getHintRight}
\Xi Board \\
absCoord? : AbsoluteCoord \\
absCoord1? : AbsoluteCoord \\
output! : Hint
\where
\IF absCoord?.y = absCoord1?.y \\
\THEN output! = EAST \\
\ELSE \\ \t1
\IF absCoord?.y < absCoord1.y \\ \t1
\THEN \exists hint: \{SOUTH, EAST, SOUTH\_EAST\} | output! = hint \\ \t1
\ELSE \exists hint: \{NORTH, EAST, NORTH\_EAST\} | output! = hint
\end{schema}

Rekening houdende met de invariant...
\begin{schema}{getValidTiles}
\Xi Board \\
output! : (Tile \cross Tile)
\where
\exists t,t1: Tile | t \not = t1  @ \\ \t1
t.type \not = HomeTile \\ \t1
t.type \not = HintTile \\ \t1
t1.type \not = HomeTile \\ \t1
t1.type \not = HintTile \implies \\ \t1
output! = (t \cross t1)
\end{schema}

Tileswitch aanroep goed checken / kijken naar functieaanroep notifystatechange
\begin{schema}{requestTilesExchange}
\Delta Board \\
output! : ({Robot \union Null} \cross {Robot \union Null})
\where
\exists (tile \cross tile1) : getValidTiles | \\ \t1
\exists abCoord, abCoord1 : AbsoluteCoord | tiles(abCoord) = tile \\ \t2
tiles(abCoord1) = tile1 \implies \\ \t3
tiles(abCoord)' = tile1 \\ \t3
tiles(abCoord1)' = tile \\ \t3
specialExchange(tile) \\ \t3
specialExchange(tile1) \\ \t3
viewer.stateChange \\ \t3
\IF tile.occupier = Null \\ \t4
tile1.occupier \not = Null\\ \t3
\THEN moveConveyorSwitch(abCoord1) \\ \t3
\IF tile1.occupier = Null \\ \t4
tile.occupier \not = Null \\ \t3
\THEN moveConveyorSwitch(abCoord)
\end{schema}

\begin{schema}{specialExchange}
\Delta Board \\
tile? : Tile
\where
\exists r : Robot | tile.occupier = r \implies \\ \t1
controller.notifyAutoMovement(r) \\ \t1
\exists rotation : Rotation | robots(r)' = rotation 
\end{schema}


PRE: tile with absoluteCoord? is empty
\begin{schema}{moveConveyorSwitch}
\Delta Board \\
absoluteCoord? : AbsoluteCoord \\
relCoord : RelativeCoord
\where
moveConveyorSwitchSub(0,1) \\
moveConveyorSwitchSub(1,0) \\
moveConveyorSwitchSub(0,-1) \\
moveConveyorSwitchSub(-1,0) \\
\end{schema}

\begin{schema}{moveConveyorSwitchSub}
\Delta Board \\
absoluteCoord? : AbsoluteCoord \\
x? : int \\
y? : int \\
relCoord : RelativeCoord
\where
relCoord.x = x? \\
relCoord.y = y? \\
\IF tiles(addAbtoRel(absoluteCoord?, relCoord)).type = ConveyorTile \\ \t1
tiles(absoluteCoord?).occupier = Null \\ \t1
\exists r : Robot | tiles(addAbtoRel(absoluteCoord?, relCoord)).occupier = r \\
\THEN moveConveyorSwitch(addAbtoRel(absoluteCoord?, relCoord)\\ \t1
saveLocation(addAbtoRel(absoluteCoord?, relCoord), r)
\end{schema}

BEGIN OK, SAVE LOCATIONS NECESSARY
\begin{schema}{moveRequest}
\Delta Board \\
localCoords? : RelativeCoord \\
robot? : Robot \\
rotation? : Rotation \\
absoluteCoord : AbsoluteCoord \\
output! : BoardResponse
\where
\IF rotation? \not = 0 \\
\THEN response! = moveRotate(localCoords?, robot?, rotation?) \\
\ELSE response! = moveWalk(localCoords?, robot?)
\end{schema}

\begin{schema}{moveRotate}
\Delta Board \\
localCoords? : RelativeCoord \\
robot? : Robot \\
rotation? : Rotation \\
output! : BoardResponse
\where
\IF localCoords.x = 0 \\ \t1
localCoords.y = 0 \\ \t1
 \exists r : robot?.rule.possibleRotations | r = rotation?\\
\THEN output! = SUCCESS \\ \t1
robots(robot?) = (robots(robot?) + rotation?) \mod 360 \\
\ELSE output! = FAILED
\end{schema}

\begin{schema}{moveWalk}
\Delta Board
localCoords? : RelativeCoord \\
robot? : Robot \\
output! : BoardResponse
\where
\IF robot?.rule.possibleMoves(localCoords? \cross robots(robot?)) = possiblePath(robot?) \\ \t1
rotation? = 0 \\
\THEN absoluteCoord = calculateNewLocation(localCoords?, robot?)\\ \t1
\IF absoluteCoord = null \\ \t1
\THEN output! = FAILED \\ \t1
\ELSE output! = checkTile(absoluteCoord?, robot?)\\ \t2
\ELSE output! = FAILED
\end{schema}

\begin{schema}{checkTile}
\Delta Board \\
absoluteCoord? : AbsoluteCoord \\
robot? : Robot \\
output! : BoardResponse
\where
saveLocation(absoluteCoord?, robot?) \\ 
\IF tiles(absoluteCoord?).type = HomeTile \\ \t1
tiles(absoluteCoord?).target = robot? \\ 
\THEN output! = WIN \\ \
\ELSE  \\ \t1
\IF tiles(absoluteCoord?).type = ConveyorTile \\ \t1
\THEN controller.notifyAutomovement(robot?) \\ \t1
\IF tiles(absoluteCoord?).type = HintTile \\ \t1
\THEN controller.notifyHint(Board.getHint(robot?), robot?) \\ \t1
output! = SUCCESS
\end{schema}

\begin{schema}{saveLocation}
\Xi Board \\
\Delta Tile \\
absCoords? : AbsoluteCoord \\
robot? : Robot
\where
\exists t : Tile | t.occupier = robot? \implies clearTile(t)\\
put(tiles(absCoords?), robot?) \\
viewer.notifyView
\end{schema}

\begin{schema}{clearTile}
\Delta Tile \\
\where
occupier' = Null
\end{schema}

\begin{schema}{put}
\Delta Tile \\
input? : Robot
\where
occupier' = input?
\end{schema}

This one should be done. Gets the path for the move it wants to do. Gets coordinate, then for all tiles it holds that if he can't pass, a conveyor tile should be in front of the obstruction (the first conveyor). Else null
\begin{schema}{calculateNewLocation}
\Xi Board \\
localCoords? : RelativeCoord \\
robot? : Robot \\
absoluteCoord : AbsoluteCoord \\
output! : (AbsoluteCoord \union Null) 
\where
\IF checkPath(localCoords?, robot?) \\
\THEN absoluteCoord = firstConveyor(c, path) \\ \t1
\exists s : ConveyorDest | s.input? = absoluteCoord @ \\ \t2
output! = s.output! \\
\ELSE output! = Null
\end{schema}

\begin{schema}{checkPath}
\Xi Board \\
localCoords? : RelativeCoord \\
robot? : Robot \\
path : seq RelativeCoord \\
absoluteCoord : AbsoluteCoord \\
bool : \bool \\
output! : \bool
\where
bool = false \\
path = robot?.rule.possibleMoves(localCoords? \cross robots(robot?)) \\
\exists (c, t) : tiles | t.occupier = robot? \implies \\ \t1
\forall (int, coord) : path | \\ \t2
((addAbtoRel(c, coord) = Null \vee \\ \t2
(tiles(addAbtoRel(c, coord).type = BrokenRobotTile) \vee \\ \t2
\exists r: Robot | tiles(addAbtoRel(c, coord).occupier = r )\implies \\ \t3
\exists (int1, coord1) : path | int1 < int \\ \t3
tiles(addAbtoRel(c, coord1).type = ConveyorTile \\ \t3
absoluteCoord = addAbtoRel(c, coord1)) \implies \\ \t4
bool = \true \\ \t4
output! = bool
\end{schema}


seq of RelativeCoord should contain conveyorTile which is not blocked (no brokenTiles / occupied tiles on path in front of conveyorTile)
Specifies that, for a given path with conveyor tile, there exists such a conveyor tile that the int is minimal. This is the AbsoluteCoord given back by the function
\begin{schema}{firstConveyor}
\Xi Board \\
firstConveyor == (AbsoluteCoord \cross seq RelativeCoord) \rightarrow AbsoluteCoord
\where
\exists absoluteCoord : AbsoluteCoord |  \\ \t1
\exists (int, coord): seq RelativeCoord |  \\ \t2
(tiles(addAbtoRel(absoluteCoord, coord).type = conveyorTile \\ \t2
conveyorInPath(absoluteCoord, (int, coord)) = addAbtoRel(absoluteCoord, coord)) \implies \\ \t3
\forall (int1, coord1): seq RelativeCoord | \\ \t3
\IF tiles(addAbtoRel(absoluteCoord, coord1).type = conveyorTile \\ \t3
\THEN int <= int1
\end{schema}

DONE
\begin{axdef}
addAbtoRel == (AbsoluteCoord \cross RelativeCoord) \rightarrow (AbsoluteCoord \union Null)
\where
\forall (a,b) : AbsoluteCoord \cross RelativeCoord | \\ \t1
\IF (0 <= a.x + b.x < BoardWidth \\ \t2
0 <= a.x + b.x < BoardHeigth) \\ \t1
\THEN
addAbtoRel(a,b).x = a.x + b.x \\ \t2
addAbtoRel(a,b).y = a.y + b.y \\ \t1
\ELSE addAbtoRel(a,b) = Null
\end{axdef}


BEGIN CONTROLLER

meerdere viewers mogelijk?
\begin{schema}{addViewer}
\Delta Controller \\
viewer? : Viewer \\
output! : Controller 
\where
\IF viewer = Null \\
\THEN viewer' = viewer? \\ \t1
output! = self \\
\ELSE output! = Null
\end{schema}


geeft door aan robot dat die fucked is
\begin{schema}{notifyAutoMovement}
\Xi Controller \\
robot? : Robot
\where
robot?.notifyAutomovement
\end{schema}

Onderscheid tussen moveRequest board en controller..
\begin{schema}{moveRequest}
\Xi Controller \\
\Delta Board \\
localCoords? : RelativeCoord \\
robot? : Robot \\
rotation? : Rotation \\
ouput! : BoardResponse
\where
\exists m : Board.moveRequest | m.localCoords? = localCoords \\ \t1
m.robot? = robot? \\ \t1
m.rotation? = rotation? \\ \t1
m.output! = output! \\ \t1
\IF output! = WIN \\ \t1
\THEN \forall r : Robot | r.terminate \\ \t2
\exists n : viewer.notifyGameOver | n.robot? = robot? \\ \t2
board.canReset \\ \t2
terminate
\end{schema}

onderscheid tussen robot notifyHint en die van controller
\begin{schema}{notifyHint}
\Xi Controller \\
\Delta Robot \\
hint? : Hint \\
robot? : Robot
\where
\exists n : robot.notifyHint | n.hint? = hint?
\end{schema}

\begin{schema}{notifyView}
\Xi Controller \\
\Delta Viewer
\where
\exists n : notifyStateChange
\end{schema}

\begin{schema}{preInitialize}
\Delta Controller
\where
board' = Null \\
robots' = Null \\
viewer' = Null
\end{schema}

\begin{schema}{postInitialize}
\Delta Controller \\
board? : Board \\
robots? : \power Robot 
\where
board' = board \\
robots' = robots?
\end{schema}

\begin{schema}{removeViewer}
\Delta Controller 
\where
viewer' = Null \\
robots' = robots \\
board' = board
\end{schema}

\begin{schema}{requestBoardStatus}
\Xi Controller \\
output! : BoardSnapShot
\where
\exists r : requestSnapShot | r.output! = output!
\end{schema}


BEGIN ROBOT

initialize geen rules nodig?
\begin{schema}{initialize}
\Delta Robot \\
controller? : Controller \\
rules? : Rules
\where
controller' = controller? \\
rules' = rules? \\
hint' = \{Null\}
\end{schema}

deze functie doet wist in principe eventuele hints, robot heeft dus een hint nodig lijkt me..
\begin{schema}{notifyAutoMovement}
\Delta Robot
\where
hint' = \{Null\}
\end{schema}

ook hier hint heel handig
\begin{schema}{notifyHint}
\Delta Robot \\
hint? : Hint
\where
hint' = \{hint?\} \union hint
\end{schema}

VIEWER

\begin{schema}{initialize}
\Delta Viewer
\where
\exists a : addViewer | a.viewer? = self \implies \\ \t1
a.output! = controller' \\ \t1
boardChanged' = \false
\end{schema}

\begin{schema}{notifyGameOver}
\Delta Viewer \\
robot? : Robot
\where
\end{schema}

\begin{schema}{notifyStateChange}
\Delta Viewer 
\where
boardChanged' = \true
\end{schema}

\begin{schema}{updateView}
\Delta Viewer
\where
\IF boardChanged = \true \\
\THEN \exists r: requestBoardStatus | output! = snapShot' \\ \t1
boardChanged' = \false \\ \t1
controller' = controller \\ 
\ELSE snapShot' = snapShot \\ \t1
boardChanged' = \false \\ \t1
controller' = controller
\end{schema}

\end{document} 
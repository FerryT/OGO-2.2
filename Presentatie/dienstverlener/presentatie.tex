\documentclass[handout,notes=only]{beamer}

\usepackage{msctexen,graphviz,graphicx,xmpmulti,tikz,alltt,movie15}
\usepackage{textpos}
\setlength{\TPHorizModule}{1cm}
\setlength{\TPVertModule}{1cm}


\title{xTremeRobotGames\\\small OGO 2.2 project}
\author{Tim van Dalen, Tony Nan, Ferry Timmers,\\
                Lasse Blaauwbroek, Femke Jansen,\\
                Jeroen Peters and Sander Breukink\\}

\institute[TU/e]{}

\AtBeginSection[]
{
  \begin{frame}
    \frametitle{Inhoudsopgave}
    \tableofcontents[currentsection]
  \end{frame}
}

\begin{document}
	\frame{\titlepage}
	
	\begin{frame}
	    \frametitle{Inhoudsopgave}
    		\tableofcontents
  	\end{frame}
	\note{
		Aan de hand van een deel van de programmaflow gaan we ons model uitleggen.
	}

	\section{Use cases}
	\begin{frame}
		\frametitle{Use cases}
		\framesubtitle{Gebruik van het systeem}
		{\small
			{\normalsize
\begin{tabbing}
Robot \= move \= Request \\
- Pre-conditions: The game is initiated.\\
- Trigger: A robot requests to move his position.\\
- Guarantee: The robot request is checked.\\
- Main Scenarios\\
\>(a) The robot requests the controller if it may move\\
\>(b) The controller sends the request to the board\\
\>(c) The board checks the request\\
- Alternatives\\
\>none\\
\end{tabbing}
}
		}
	\end{frame}
	\note{
		Niet zo heel boeiend, even programmaflow uitleggen.
	}

    \begin{frame}
		\frametitle{Use cases}
		\framesubtitle{Some use cases for the game}
		{\small
			{\footnotesize
\begin{tabbing}
Return \= Hint \= tile \\
- Pre-conditions: The game is initiated, the indicated position\\
\>is a hint tile and the move request has been approved.\\
- Trigger: A robot requests to move his position to a hint tile.\\
- Guarantee: The robot is moved to the indicated position which\\
\>is a hint tile and a hint is given to the robot.\\
- Main Scenarios\\
\>(a) The robot is moved to the indicated position, which is a hint tile\\
\>(b) The board notifies the controller that the move request has been approved\\
\>(c) The controller notifies the robot that his move request has been approved\\
\>\> and that the robot has been moved to the requested position\\
\>(d) The board generates a hint for the robot\\
\>(e) The board notifies the controller of the hint\\
\>(f) The controller sends the hint to the robot\\
- Alternatives\\
\>none\\
\end{tabbing}
}
		}
	\end{frame}

    \begin{frame}
		\frametitle{Use cases}
		\framesubtitle{Some use cases for the game}
		{\small
			\begin{alltt}

Update viewer
- Pre-conditions: The game is initiated.
- Trigger: The viewer receives a notify along with a new snapshot from the controller.
- Guarantee: The viewer has been updated.
- Main Scenarios
    (a) The board makes a snapshot
    (b) The board sends the snapshot to the controller
    (c) The controller sends the snapshot to the viewer
    (b) The viewer updates the view with the received snapshot
- Alternatives
    none
    
\end{alltt}
		}
	\end{frame}

	\section{Message Sequence Charts}
	\subsection{High level}
	\begin{frame}
		\frametitle{High level Message Sequence Chart}
		\framesubtitle{De flow van het spel}
		\digraph[scale=0.4]{HMSC}{
begin2 [label="",shape="invtriangle"];
end2 [label="",shape="triangle"];
initview [label="Initialize viewer",shape="Mrecord"];
updateview [label="Update viewer",shape="Mrecord"];
begin2->initview;
initview->updateview;
updateview->updateview;
updateview->end2;
begin [label="",shape="invtriangle"];
end [label="",shape="triangle"];
p1 [label="",shape="point"];
p2 [label="",shape="point"];
p22 [label="",shape="point"];
p3 [label="",shape="point"];
p4 [label="",shape="point"];
p5 [label="",shape="point"];
p55 [label="",shape="point"];
p6 [label="",shape="point"];
p66 [label="",shape="point"];
p7 [label="",shape="point"];
p77 [label="",shape="point"];
p8 [label="",shape="point"];
init [label="Initialize",shape="Mrecord"];
mvreq [label="Robot move request",shape="Mrecord"];
mvrej [label="Reject move",shape="Mrecord"];
retnt [label="Return Normal tile",shape="Mrecord"];
retht [label="Return Hint tile",shape="Mrecord"];
retct [label="Return Conveyor tile",shape="Mrecord"];
retmt [label="Return Home tile",shape="Mrecord"];
ordex [label="Ordinary exchange",shape="Mrecord"];
spcex [label="Special exchange",shape="Mrecord"];
endgame [label="End game",shape="Mrecord"];
notrob1 [label="Notify robots",shape="Mrecord"];
notrob2 [label="Notify robots",shape="Mrecord"];
begin->p1;
p1->init;
init->p2;
p2->mvreq;
mvreq->p3;
p3->mvrej;
mvrej->p2;
p3->p4;
p4->retnt;
p4->retht;
p4->retct;
p4->retmt;
retnt->p5;
retht->p5;
retct->p5;
p5->p55;
p55->p66;
p66->p6;
p55->notrob1;
notrob1->p66;
p6->ordex;
p6->spcex;
ordex->p7;
spcex->p7;
p7->p77;
p77->p22;
p22->p2 [tailport=e];
p77->notrob2;
notrob2->p22;
retmt->endgame;
endgame->p8;
p8->p1;
p8->end;
}

	\end{frame}
	\note{
		Volgorde/loops in het programma uitleggen.
	}
	
	\subsection{Use case level}

	\begin{frame}
		\frametitle{Use case level Message Sequence Charts}
		\framesubtitle{Move request}
		\scalebox{0.9}{
			\begin{msc}
msc
{

d [label="Board"],
c [label="Controller"],
a [label="Robot"],
r [label="Rule"];

d box d [label=""],
c box c [label=""],
a box a [label=""],
r box r [label=""];

|||;

a => c [label="moveRequest(localCoords, robot, rotation)"];
c => d [label="moveRequest(localCoords, robot, rotation)"];
d rbox d [label="get possible moves"], r note r [label="references the robot's ruleset"];
d rbox d [label="get possible rotations"];
d rbox d [label="check if robot can be placed"];

|||;

a box a [label="", textbgcolor="black"],
c box c [label="", textbgcolor="black"],
d box d [label="", textbgcolor="black"],
r box r [label="", textbgcolor="black"];

}
\end{msc}

		}
		\digraph[scale=0.3]{HMSC_req}{
rankdir=LR;
p2 [label="",shape="point"];
p3 [label="",shape="point"];
p4 [label="",shape="point"];
init [label="Initialize",shape="Mrecord"];
mvreq [label="Robot move request",shape="Mrecord",style=filled];
mvrej [label="Reject move",shape="Mrecord"];
retnt [label="Return Normal tile",shape="Mrecord"];
retht [label="Return Hint tile",shape="Mrecord"];
retct [label="Return Conveyor tile",shape="Mrecord"];
retmt [label="Return Home tile",shape="Mrecord"];
init->p2;
p2->mvreq;
mvreq->p3;
p3->mvrej;
mvrej->p2;
p3->p4;
p4->retnt;
p4->retht;
p4->retct;
p4->retmt;
}
	\end{frame}
	\note{
		Uitleggen dat het grootste probleem dat we gehad hebben is hoe we messages doorsturen.
	}
	\begin{frame}
		\frametitle{Use case level Message Sequence Charts}
		\framesubtitle{Hint move}
		\scalebox{0.8}{
			\begin{msc}
msc
{

d [label="Board"],
c [label="Controller"],
a [label="Robot"];

d box d [label=""],
c box c [label=""],
a box a [label=""];

|||;

d rbox d [label="calculateNewLocation(localCoords, robot)"];
d rbox d [label="saveLocation(absCoords, robot)"];
d >> c [label="SUCCESS"];
c >> a [label="SUCCESS"];
d rbox d [label="getHint(robot)"];
d -> c [label="notifyHint(hint, robot)"];
c -> a [label="notifyHint(hint)"];

|||;

a box a [label="", textbgcolor="black"],
c box c [label="", textbgcolor="black"],
d box d [label="", textbgcolor="black"];
}
\end{msc}
		}
		\digraph[scale=0.3]{HMSC_mvht}{
rankdir=LR;
p3 [label="",shape="point"];
p4 [label="",shape="point"];
p5 [label="Update viewer",shape="Mrecord"];
p6 [label="",shape="point"];
notrob1 [label="Notify robots",shape="Mrecord"];
p55 [label="",shape="point"];
p66 [label="",shape="point"];
mvreq [label="Robot move request",shape="Mrecord"];
mvrej [label="Reject move",shape="Mrecord"];
retht [label="Return Hint tile",shape="Mrecord",style=filled];
mvreq->p3;
p3->mvrej;
p3->p4;
p4->retht;
retht->p5;
p5->p55;
}

	\end{frame}
	\begin{frame}
		\frametitle{Use case level Message Sequence Charts}
		\framesubtitle{Viewer update}
		\scalebox{0.8}{
			\begin{msc}
msc {

b [label="Board"],
c [label="Controller"],
d [label="Viewer"];

b box b [label=""],
d box d [label=""],
c box c [label=""];

|||;

b rbox b [label="make snapshot"];
b -> c [label="notifyViewer(snapshot)"];
c -> d [label="notifyStateChange(snapshot)"];
d rbox d [label="updateView()"];

|||;

d box d [label="", textbgcolor="black"],
c box c [label="", textbgcolor="black"],
b box b [label="", textbgcolor="black"];

}
\end{msc}
		}
		\digraph[scale=0.3]{HMSC_upview1}{
rankdir=LR;
p5 [label="Update viewer",shape="Mrecord",style=filled];
p6 [label="",shape="point"];
retnt [label="Return Normal tile",shape="Mrecord"];
retct [label="Return Conveyor tile",shape="Mrecord"];
retht [label="Return Hint tile",shape="Mrecord"];
ordex [label="Ordinary exchange",shape="Mrecord"];
spcex [label="Special exchange",shape="Mrecord"];
retnt->p5;
retct->p5;
retht->p5;
p5->p6;
p6->ordex;
p6->spcex;
} 
	\end{frame}

	\section{Class diagram}
	\begin{frame}
		\frametitle{Class diagram}
		\framesubtitle{Hoe de onderdelen samenwerken}
		\let\l=\relax
\let\<=\relax
\let\>=\relax
\digraph[scale=.5]{classdiagram}{
margin=0
fontsize=8
fontname=Helvetica
compound=true
splines=ortho
node [fontsize=8, fontname=Helvetica, shape=record]
edge [fontsize=8, fontname=Helvetica, arrowhead=open, labeldistance=2]
Board [label="{Board|- width : int\l- height : int\l|+ canReset() : bool\l+ initialize()\l+ moveRequest(loc : RelativeCoord, r : Robot, rot : Rotation) : BoardResponse\l+ requestSnapshot() : BoardSnapshot\l+ requestTilesExchange() : bool\l- getHint(r : Robot) : Hint\l- calculateNewLocation(loc : RelativeCoord, r : Robot) :
AbsoluteCoord\l- getValidTiles() : TilePair\l- reset()\l- saveLocation(loc : AbsoluteCoord, r : Robot)\l}"]
Hint [label="{[Hint]| NORTH\l NORTH_EAST\l EAST\l SOUTH_EAST\l SOUTH\l SOUTH_WEST\l WEST\l NORTH_WEST\l}"]
BoardResponse [label="{[BoardResponse]| FAILED\l SUCCESS\l WIN\l}"]
Viewer [label="{Viewer||+ initialize()\l+ notifyGameOver(r : Robot)\l+ notifyStateChange()\l- updateView()\l}"]
BoardSnapshot [label="{BoardSnapshot||}"]
Controller [label="{Controller||+ addViewer(v : Viewer) : Controller\l+ moveRequest(loc : RelativeCoord, r : Robot, rot : Rotation) : BoardResponse\l+ notifyAutoMovement(r : Robot)\l+ notifyHint(h : Hint, r : Robot)\l+ notifyView()\l+ preInitialize()\l+ postInitialize(b : Board, rs : RobotList)\l+ removeViewer()\l+ requestBoardSnapshot() : BoardSnapshot\l- terminate()\l}"]
AbsoluteCoord [label="{AbsoluteCoord|+ x : int\l+ y : int\l|}"]
Tile [label="{Tile||}"]
/**/
subgraph cluster_Tiles {
NormalTile [label="{NormalTile||}"]
HomeTile [label="{HomeTile||}"]
HintTile [label="{HintTile||}"]
ConveyorTile [label="{ConveyorTile|- rot : Rotation|}"]
BrokenRobotTile [label="{BrokenRobotTile||}"]
}
/**/
Robot [label="{Robot||+ initialize(c : Controller, r : Rule)\l+ notifyAutoMovement()\l+ notifyHint(h : Hint)\l+ terminate()\l}"]
RelativeCoord [label="{RelativeCoord|+ x : int\l+ y : int\l|}"]
Rule [label="{\<\<Rule\>\>|- possibleMoves : RelativeCoordList\l- possibleRotations : RotataionList\l|}"]
Rotation [label="{[Rotation]| 0DEG\l 90DEG\l 180DEG\l 270DEG\l}"]
/**/
Board->Controller [taillabel=1, headlabel="0..*"]
Board->Tile [arrowtail=diamond,dir=both, taillabel=1,headlabel="*"]
Board->Robot [taillabel=1, headlabel="0..*"]
/**/
Controller->Viewer [taillabel=1, headlabel=1, arrowhead=none]
Controller->Robot [taillabel=1, headlabel="*", arrowhead=none]
/**/
Tile->Robot [taillabel=1, headlabel="              0..1 - occupier"]
/**/
HomeTile->Robot [taillabel=1, headlabel="              1 - homeRobot"]
/**/
Robot->Rule [taillabel="*", headlabel=1]
/**/
BoardSnapshot->Tile [taillabel=1, headlabel="*"]
/**/
NormalTile->Tile [ltail=cluster_Tiles,arrowhead=empty]
/**/
Board->Hint [style=dashed]
Board->BoardResponse [style=dashed]
Board->AbsoluteCoord [style=dashed]
Viewer->BoardSnapshot [style=dashed]
Controller->AbsoluteCoord [style=dashed]
Robot->RelativeCoord [style=dashed]
Rule->Rotation [style=dashed]
ConveyorTile->Rotation [style=dashed]
} 
	\end{frame}
	\note{
		Kort aangeven wie de baas is etc.
	}

	\section{Test cases}
	\begin{frame}
		\frametitle{Test cases}
		\framesubtitle{De specificatie testen}
		\begin{itemize}
			\item<1,3,5>{Ingesloten robot}
			\item<3,5>{Ingesloten home tile}
		\end{itemize}
		
		\only<2>{\begin{textblock}{5}(0, -4)\includegraphics[width=0.8\paperheight]{frame-testcase1.png}\end{textblock}}
		\only<4>{\begin{textblock}{5}(0, -4)\includegraphics[width=0.8\paperheight]{frame-testcase2.png}\end{textblock}}
	\end{frame}
	\note{
		In ons spel is het heel belangrijk dat elke robot ten alle tijde zijn hometile kan bereiken, anders zou dat niet eerlijk zijn voor robots die dat niet kunnen. Daarom hebben wij een functie gemaakt die kijkt of dit ook werkelijk zo is, om ervoor te zorgen dat situaties waarbij robots hun hometile niet kunnen bereiken niet voorkomen. Deze functie definieert paden tussen tegels, rekening houdende met obstakels als Conveyor tiles en Broken Robot tiles. Als een robot zijn hometile kan bereiken, betekent dat in dit geval simpelweg dat er een pad is van de robot naar zijn hometile (omgekeerd hoeft dat niet het geval te zijn). Omdat dit heel belangrijk is in ons spel en de stakeholder ook meerdere testcases hiervan heeft aangeleverd hebben wij besloten deze functie te testen en zullen nu 2 testcases laten zien.
	}
	\note{
		Testcase 1: In deze testcase tekenen wij het bord, met hierop een robot en bijbehorende hometile. Echter, deze robot is ingesloten door BrokenRobot tiles en kan zijn hometile dus nooit bereiken. (Zie plaatje, gemaakt door onze viewer). Deze situatie moet dus onmogelijk zijn, onze check functie moet false retourneren (en dat gebeurt ook).
		}
	\note{ 
		Testcase 2: In deze testcase tekenen wij een bord met hierop een robot en bijbehorende hometile. De hometile is ingesloten door BrokenRobot tiles, behalve 1 tile, dat is een conveyor tile. De robot is niet ingesloten, maar staat ernaast (zie plaatje). De robot kan via de Conveyor tile, die richting de hometile gaat, nog steeds zijn hometile bereiken. De check functie moet dus true retourneren (en dat gebeurt ook). Deze testcase dient er vooral voor om te checken of onze check functie ook rekening houdt met Conveyor tiles en ze niet alleen ziet als (mogelijk) obstakel.
	}

	\section{Implementatie}
	\begin{frame}
		\frametitle{Implementatie}
		\framesubtitle{Het werkende programma}
		\begin{itemize}[<+->]
                        \item Hoofdprogramma
                        \item Viewers
                \end{itemize}
	\end{frame}
	\note{
		Hoofdprogramma in java geimplementeerd, meerdere mogelijke viewer ontwikkeld (\'e\'en tegelijk). Hier laten we er \'e\'en zien die een top-down video genereert.
	}
	\subsection{Game verslag}
	\begin{frame}
		\includemovie[poster,autoplay]{10cm}{10cm}{movie.avi}
	\end{frame}

	\begin{frame}

	\end{frame}
\end{document}

\documentclass[a4paper,11pt]{article}
\usepackage{graphicx,listings,a4wide}
%\usepackage[firstpage]{draftwatermark}
%\SetWatermarkLightness{0.5}
%\SetWatermarkScale{4}
\setcounter{tocdepth}{2}

\newcommand{\question}[2]{\medskip\par\noindent\textbf{#1}\\\hangindent=0.5cm#2}

\title{Report on OGO 2.2 \\ Software specification\\ Testcases for group 3}
\author{
        Tim van Dalen, Tony Nan, Ferry Timmers, \\ Lasse Blaauwbroek, Femke Jansen, \\Jeroen Peters and Sander Breukink\\ OGO 2.2 group 6 \\
                Department of Computer Science\\
        Technical University Eindhoven\\
}
\date{\today}

\begin{document}

\maketitle

    \begin{abstract}
    This document contains several non-trivial use cases, based upon the formal specification of group 6. The use cases are each informally described with an input and expected output. ToDo: beschrijven input alleen specifiek die entiteiten waarvan de settings anders zijn.
    \end{abstract}
	
	\section{0 Pieces}
    Input: each player has 0 pieces.\\
    Output: error, each player has to begin with at least 1 piece.

	\section{Fox eats dolphin in water}
    Input: a fox is 2 tiles away from the land and there is a dolphin between the fox and the land. The fox moves towards the land and onto the tile that is occupied by the dolphin.\\
    Output: despite of being in the water, the fox will still eat the dolphin.\\

    \section{Fox eats dolphin at bridge}
    Input: a bridge has been flooded, and while it was flooded a dolphin swam to one end of it, but the tide lowered and the dolphin can't move anymore. A fox goes up to the bridge from the other side.\\
    Output: because the fox will go to the same tile as the dolphin is currently, the fox will eat the dolphin.\\

    \section{Fox wants to go into the water}
    Input: a fox is currently on the land but wants to go into the water.\\
    Output: because the fox $\leq$ 2 tiles away from land it can still move and thus go into the water. If there is a dolphin on the water tile, the fox kills the dolphin.\\

    \section{A player has won}
    Input: player 1 has eliminated the last animal of player 2.\\
    Output: player 1 has won and the game ends.\\

    \section{Fox doesn't drown}
    Input: a silly fox goes into the water and goes in to deep and can't move anymore.\\
    output: no matter how long it takes for the tide to change back, the fox stays alive unless it gets eaten by a dolphin.\\


    Note: because we want to keep it a fair game, all testcases for foxes are also valid for the dolphin in the other way around.\\
\end{document} 
\documentclass[12pt]{article}
\usepackage{ltcadiz}
\usepackage{listings}
\usepackage{amssymb}
\lstset{breaklines=true, numbers=left}

\title{Report on OGO 2.2 Softwarespecification\\ Assignment 1b}
\author{
        Tim van Dalen, Tony Nan, Ferry Timmers and Sander Breukink\\ OGO 2.2 group 6 \\
                Department of Computer Science\\
        Technical University Eindhoven\\
}
\date{\today}

\begin{document}

\maketitle

\begin{abstract}
This document presents a formal specification and implementation of assignment 1b of OGO 2.2 Softwarespecification. Also included are the informal requirement of the stakeholder, their judgement about the formal specification and testcases for the implementation.
\end{abstract}

\section{Informal requirements}
\paragraph{Input} Provided is a simple (ASCII) file with a few numbers separated by a space of arbitrary size. There is no limit to the size of the file, but there has to be at least one number in the file.

\paragraph{Output} Take all unique numbers from the file of the form $n^3$ where $n$ any natural number and select the lower median of these numbers. A number is of the form $n^3$, when the cube root of the number is a natural number.

We can find the lower median of a sequence of numbers by sorting the sequence and selecting the number in the middle of the sequence. If the size of the sequence is even, the lower median is the left element of the two elements in the middle.

\section{Formal specification}

This section describes a formal specification adhering to the informal requirements described above.

We first have to define an enumeration \textit{Cubes} that contains all sets of natural numbers for which each element is a cube of a natural number.
\begin{axdef}
Cubes == \{ x : \power \nat | (\forall y : x | (\exists z : \nat @ y^3 = z) )\}
\end{axdef}

We define a function that calculates the lower median of a set of natural numbers. It defines two disjoint subsets $a$ and $b$ that are the two (almost) equal halves of the input. All elements in $a$ are smaller than $b$, and the output is the number that is wedged between the two halves. The union of $a$, $b$ and the output is the whole input.
\begin{schema}{Median}
s? : \power \nat \\
m! : \nat
\where
\exists a,b : \power s? | [ \: a \union b \union m! = s? \\
                          \t1 a \cap b = \emptyset \\
                          \t1 m! \notin ( a \union b ) \\
                          \t1 \# b - \# a = 0 \: \vee \# b - \# a = 1 \\
                          \t1 \forall x : a , y : b | x < m! < y \: ]
\end{schema}

Now we define the input and output of our specification. The output is computed by taking a set from \textsl{Cubes} with only cubes from the input (and no other cubes) and computing the median from it. The input sequence must contain at least one element.
\begin{axdef}
Input : \seq \nat
\where
\#Input > 0
\end{axdef}

\begin{axdef}
Output : \nat
\where
\exists X : Cubes | [ \: X \subseteq \ran Input \\ \t1 \forall Y : \ran Input \setminus X | Y \not\subseteq Cubes \\ \t1 Output = Median (X) \: ]
\end{axdef}

\section{Judgement of the formal specification}
The formal specification seems to be completely correct. We only have some comments on the style and explanations. We would rather have seen the definition of \textsl{Cubes} to be renamed to something like \textsl{PowerCubes}. \textsl{Cubes} implies it is a set of cube integers while it actually is the powerset of the set of cube integers.

We would also rather have seen a schema instead of two axiomatic definition for the input and output. When a schema is used, it looks more like an algorithm, in our opinion. The explanations of the schema's are not always very easy to understand but we acknowledge the difficulty of the subject and do not think it very easy to improve.

\section{Testcases}

\paragraph{Double one testcase}
This case checks if the output is still correct when the input is 5 times the same number. \\
Input: \{1 1 1 1 1\} \\
Expected output: 1 (test succeeded)

\paragraph{Inverse order}
This case checks whether the algorithm works if the order is inverted. \\
Input: \{1000 512 216 64 8 0\}
Expected output: 64 (test succeeded)

\paragraph{Invalid numbers}
This case checks if it ignores values which are not $n^3$. \\
Input \{-3 -2,5 0 8 64,001 65 216 513 1000\} \\
Expected output: 8 (test succeeded)

\paragraph{Negative input}
This case checks if it handles values below 0 correctly. \\
Input \{-216 -64 -8 0 1 8\} \\
Expected output: -8 (test succeeded)

\paragraph{Combination}
In this test case there are double values, they aren't ordered and there are number which are not $n^3$.\\
Input: \{1 5 -8 -7 -6,5 -1 -8 2 0,5 3\} \\
Expected output: -1 (test succeeded)

\section{Implementation}
We chose Java to implement the specification because that's the language that everyone in our group has the most experience with. The implementation follows our specification correctly. If the input is empty, the program will throw an exception and terminate abnormally.

First, in line 28 the input's cubed root is calculated. Then in lines 30-34 it adds the number to the array if there exists an $n \in \mathbb{N}$ for which the input number is equal to $n^3$, all numbers that satisfy this condition are added to the arraylist. If no results were added an exception is thrown to show that the (usable) input was empty.

In lines 42-51 we sort the arrayList and remove double values to make sure that we only pick the unique numbers from the input.

In line 53-55 we pick out the lower median and return the final result.

All testcases resulted in a succes. The method implementing the algorithm is shown in appendix \ref{codelisting} and requires an import for \textsl{java.io.InputStream}, \textsl{java.util.ArrayList}, \textsl{java.util.Collections} and \textsl{java.util.Scanner}.

\appendix

\section{Java implementation listing}
\label{codelisting}
\begin{lstlisting}[language=java]
/**
 * This method first calculates the cube root of all
 * the natural numbers on the input, then it puts all
 * of them in an arraylist, sorts the arraylist,
 * removes non-unique values and picks out the lower
 * median, this lower median is returned.
 * @throws IllegalArgumentException When a number in the
 * sequince doens't have a natural number as it's cube
 * root
 * @throws IllegalArgumentException When input is empty
 * @param stream The input stream with integers
 * disjunct bij spaces
 * @return The lower mean from the cubed roots of the
 * unique input integers
 */
public static int FindLowMed(InputStream stream){
    Scanner scanner = new Scanner (stream);
    ArrayList<Integer> n = new ArrayList<Integer>();
    int lowerMedian;
    double input;
    int result;

    while (scanner.hasNextDouble()){
        //only integers matter to us
        input = scanner.nextDouble();
        //round for approaching the root
        //(for example 1000^(1/3) = 9,999999999 -> 10)
        result = (int) Math.round(Math.cbrt(input));

        if(Math.pow(result,3) == input){
            //ignore for which there doesn't exist an n^3 (n is a natural
            //number) which is equal to it
            n.add((int) input);
        }
    }

    if(n.isEmpty()){
        throw new IllegalArgumentException(
            "incorrect input: (usable) input is empty");
    }

    //sort the arraylist
    Collections.sort(n);

    for(int i = 0; i < n.size() - 1; i++){
        //remove all non-unique values
        if(n.get(i) == n.get(i+1)){
            n.remove(i+1);
            i--;
        }
    }

    lowerMedian = n.get((int) (n.size()-1)/2);

    return lowerMedian;
}
\end{lstlisting}

\end{document}

\documentclass[12pt]{article}
\usepackage{ltcadiz}
\usepackage{listings}
\lstset{breaklines=true, numbers=left}

\title{Report on OGO 2.2 Softwarespecification assignment 1a}
\author{
        Femke Jansen and Lasse Blaauwbroek OGO 2.2 group 6 \\
                Department of Computer Science\\
        Technical University Eindhoven\\
}
\date{\today}

\begin{document}

\maketitle

\begin{abstract}
This document presents a formal specification and implementation of assignment 1a of OGO 2.2 Softwarespecification. Also included are the informal requirement of the stakeholder, their judgement about the formal specification and testcases for the implementation.
\end{abstract}

\section{Informal requirements}
\paragraph{Input} A simple textfile consisting of real numbers separated with whitespace.
\paragraph{Output} Given is a sequence of numbers from which some numbers needs to be summed. The indices of the numbers that need to be added are given by the Fibonacci sequence. If this indice exists twice in the fibonacci sequence, this number should be summed twice. The output is the result of the sum in flat text.
\paragraph{Example}
Input: \{-5.0 43.0 25.0 -83.0 67.0 -97.0\}\\
The input has $6$ elements, so the following part of the fibonacci sequence is used: $\{0, 1, 1, 2, 3, 5, 8, 13, 21, ...\}$. Using the sequence as indices of the numbers on input, we get the following numbers that need to be summed: \\ \{-5.0, 43.5, 43.5, 25, -83.0, -97.0\}. \\ The summation of these numbers is $73$.

\section{Formal specification}

This section describes a formal specification adhering to the informal requirements described above.

We first have to define a function  \textit{fib} that returns the $n$'th Fibonacci number. Fibonacci numbers are recursively defined using the two previous Fibonacci numbers.
\begin{axdef}
fib : \nat \fun \nat
\where
fib(0) = 0 \\
fib(1) = 1 \\
\forall n : \nat | n \geq 2 @ fib(n) = fib(n-2) + fib(n-1)
\end{axdef}

Next, we also need a function \textit{sum} that returns the summation of a given sequence of numbers. This function is recursively defined as well, by computing the sum of the tail of the sequence until the sequence is empty, and adding all of the heads to this sum.
\begin{axdef}
sum : \power (\seq \real \fun \real)
\where
sum(\langle\rangle) = 0 \\
\forall (x) : \seq \real | \#x > 0 @ sum(x) = sum(\tail x) + \head x
\end{axdef}

Last but not least, we need a method that filters certain numbers from a given sequence and places those numbers in a new sequence. This filtering method selects those numbers from the given sequence with an index equal to a Fibonacci number. The idea is that we can use the new sequence by summing it to calculate the output. The details are a bit tricky, because sequences begin numbering with one, and the Fibonacci sequence starts with zero. Therefore, we increment each Fibonacci number with one, and decrement each sequence index by one to get a correct mapping from Fibonacci numbers to indices. The $z$ variable in the specification below filters the right elements from the given sequence. It realizes this by ensuring that the $z$'th element in the new sequence is equal to the element in the first sequence with as index the $z$'th Fibonacci number.
\begin{axdef}
filterFibIndices : \power (\seq \real \fun \seq \real)
\where
\forall (x,y) : filterFibIndices; z :\nat | \\
\t1 1 \leq z \wedge z \leq \# y \wedge fib(z-1)+1 \leq \#x @ \\
\t1 y(z) = x(fib(z-1)+1)
\end{axdef}

The final algorithm to take the summation of all numbers with a Fibonacci index is now very straightforward and given below.
\begin{schema}{sumFibIndices}
input? : \seq \real \\
output! : \real
\where
output! = sum(filterFibIndices(input?))
\end{schema}

\section{Judgement of the formal specification}

\section{Testcases}

\paragraph{Null testcase}
This case checks whether an empty input creates the correct output. \\
Input: \{\} \\
Expected output: 0 (test succeeded).

\paragraph{Double one testcase}
This case checks whether the number on index one is summed twice. \\
Input: \{0 1\} \\
Expected output: 2 (test succeeded)

\paragraph{Negative numbers}
This case checks whether negative numbers are summed correctly. \\
Input: \{-3 -2 -5 -8 -10\} \\
Expected output: -20 (test succeeded)

\paragraph{Large mixed list}
This case check whether a longer list with mixed positive, negative and non-integer numbers are summed correctly. \\
Input \{-800 2,85 -85,15 -7482,2654 87 0,002 -0,008 784 -235 12\} \\
Expected output: -8596,7134 (test succeeded)

\section{Implementation}
We chose to implement the algorithm in the Java programming language, because everyone in our group is familiar with it. The implementation is very different from the formal specification, and consist of one loop trough all input elements.

A variable keeps track of the index of the current element, and a variable keeps track of the current Fibonacci number. As soon as the index is equal to the current Fibonacci number, the element is added to a sum variable, and the next Fibonacci number is computed. The Fibonacci numbers are computed in lines 36-38, corresponding to the \textsl{fib} function in the specification; the sum is computed on line 28, corresponding to the \textsl{sum} function of the specification; the filtering of indices is done at line 27, corresponding to the \textsl{filterFibIndices} function.

All testcases are successfully tested (see previous sections). The method implementing the algorithm is shown below and requires an import for \\ \textsl{java.io.InputStream} and \textsl{java.io.Scanner}:

\appendix

\section{Java implementation listing}

\begin{lstlisting}[language=java]
/**
 * This method is an algorithm that gives the sum of
 * the (real) numbers in a stream. But only the numbers
 * with indices equal to a fibonacci number are used.
 *
 * @param stream The stream to take the numbers from
 * @return The sum of the elements in the stream
 *         with indices equal to a fibonacci number
 */
public static double fibAlgo(InputStream stream) {
    Scanner scanner = new Scanner(stream);
    // Fibonacci is repeatedly computed using older
    // fibonacci numbers (this is more efficient than
    // calculating each fibonacci number from scratch)
    int secondOldFib = 0;
    int firstOldFib = 1;
    int fib = 0;
    double current;
    // Index represents the index of the double in the
    // stream
    int index = 0;
    double sum = 0;
    while (scanner.hasNextDouble()) {
        current = scanner.nextDouble();
        // When the index equals to the fibonacci
        // number, we add it to the sum
        if (index == fib) {
            sum += current;
            // Index one is an exception because it
            // occurs twice in the fibonacci sequence
            if (index == 1) {
                sum += current;
            }
            // Calculate the new fibonacci number and
            // store the old numbers
            fib = secondOldFib + firstOldFib;
            secondOldFib = firstOldFib;
            firstOldFib = fib;
        }
        index++;
    }
    return sum;
}
\end{lstlisting}

\end{document} 